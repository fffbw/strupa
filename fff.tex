% Template for Fridays For Future Designed Documets
% Template Jonas Klein, 2022
\documentclass[a4paper,
  ]{scrartcl}
\usepackage[
  top=0pt,
  bottom=1.5cm,
  left=1.5cm,
  right=1.5cm,
  headheight=1.5cm,
  footskip=35pt,
  headsep=20pt,
  includeheadfoot
]{geometry}
\usepackage[ngerman]{babel} 
\usepackage[T1]{fontenc}
\usepackage{lmodern}
\setlength{\parindent}{0pt}
% xelatex or lualatex needed for fontspec
% maybe the font file have to be installed
\usepackage{fontspec}
\setmainfont[Ligatures=TeX]{Jost-Regular}


% COLORED SECTIONS
\usepackage{xcolor}
\usepackage{lipsum}
\definecolor{fffgreen}{HTML}{1DA64A}
\definecolor{fffdgreen}{HTML}{1B7340}
\definecolor{ffflgreen}{HTML}{DAEFD3}

\addtokomafont{section}{\mysection}
\newcommand{\mysection}[1]{%
    \setlength{\fboxsep}{0cm}%already boxed
    
    \colorbox{fffdgreen!100}{%
        \begin{minipage}{\linewidth}%
            \vspace*{5pt}%Space before
            \color{white}
            \MakeUppercase{#1}
            \vspace*{5pt}%Space after
        \end{minipage}%
}}
% subsection

\addtokomafont{subsection}{\mysectionb}
\newcommand{\mysectionb}[1]{%
    \setlength{\fboxsep}{0cm}%already boxed
    
    \colorbox{fffgreen!100}{%
        \begin{minipage}{\linewidth}%
            \vspace*{5pt}%Space before
            \color{white}
            \MakeUppercase{#1}
            \vspace*{5pt}%Space after
        \end{minipage}%
}}

% subsubsection

\addtokomafont{subsubsection}{\mysectionc}
\newcommand{\mysectionc}[1]{%
    \setlength{\fboxsep}{0cm}%already boxed
    
    \colorbox{ffflgreen!100}{%
        \begin{minipage}{\linewidth}%
            \vspace*{2pt}%Space before
            \MakeUppercase{#1}
            \vspace*{2pt}%Space after
        \end{minipage}%
}}

% make links green
\usepackage{hyperref}
\hypersetup{
    colorlinks=true,
    linkcolor=fffgreen,
    filecolor=fffgreen,      
    urlcolor=fffgreen
}

% header and footer
\usepackage{fancyhdr}
\pagestyle{fancy}
\usepackage{overpic}
\newcommand{\custompagemark}{
  \begin{overpic}[scale=1,,tics=10]
    {pagecount.png}
    \put(10,50){\color{white}{\thepage}}
  \end{overpic}
}
\fancyhead{} % clear all header fields
\fancyfoot{} % clear all footer fields
\renewcommand{\headrulewidth}{0pt}
\fancyhead[L]{\custompagemark}
\fancyheadoffset[L]{\dimexpr\oddsidemargin+1in\relax}

% for Information symbols
\usepackage{wrapfig}
\usepackage{tabularx}
\usepackage{graphicx}
\renewcommand\tabularxcolumn[1]{m{#1}}% for vertical centering text in X column


% parallel text
\newcommand{\paralleltext}[2]{\\
\begin{tabularx}{\textwidth}{X X}
  
#1 & #2 \\
\end{tabularx}
}

\usepackage{tocloft}
\makeatletter
\renewcommand{\@cftmaketoctitle}{}
\makeatother

% for organigram
\usepackage{tikz}
\usetikzlibrary{shapes.geometric, arrows}

%%%%%%%%%%%%%%%%%%%%%%%%%%%%%%%%%%%%%%%
% Properties %%%%%%%%%%%%%%%%%%%%%%%%%%
%%%%%%%%%%%%%%%%%%%%%%%%%%%%%%%%%%%%%%%

% Comment out for no version numbers
\newcommand{\docversion}{StruPa\_BW\_v3.0 (\today)}
\fancyfoot[L]{{\color{fffgreen}\docversion}}

\begin{document}
% place logo big in the middle
%\begin{center}
%  \includegraphics[width=0.5\textwidth]{Logo.png}
%\end{center}


% logo in the left Floating parallell to text
% for compact layout without table of contents
\begin{tabularx}{\textwidth}{X m{5cm}}
      \centering \Huge \color{fffgreen} \MakeUppercase{Strukturpapier \\ von \\Fridays for Future\\Baden-Württemberg} & \includegraphics[width=5cm]{Logo.png}
\end{tabularx}



% Comment out if no table of contents needed or if you set it later manually
% \tableofcontents
%%%%%%%%%%%%%%%%%%%%%%%%%%%%%%%%%%%%%%%%%%%
% CONTENT %%%%%%%%%%%%%%%%%%%%%%%%%%%%%%%%%
%%%%%%%%%%%%%%%%%%%%%%%%%%%%%%%%%%%%%%%%%%%
\section{Grundsatz - StruPa}
Dieses Strukturpapier hat den Zweck, der Fridays for Future-Bewegung in Baden-Württemberg
Legitimation zu verschaffen. Es soll die gegenseitige vertrauensvolle und wertschätzende Umgangsform
um Transparenz und Basisdemokratie ergänzen. Die Struktur soll einen einfachen Zusammenschluss auf
Landesebene ermöglichen und jedem schnellen Zugang zu Informationen gewähren. Damit einher geht
das Ziel, möglichst flache Hierarchien aufzubauen. Ebenfalls wichtig ist es, eine Landesebene zu schaffen,
die handlungsfähig bleibt und jeder
Landesarbeitsgruppe erlaubt, effektiv und effizient zu arbeiten.
\section{Inhaltsverzeichnis}
\tableofcontents
\section{Organigramm}
Das Organigramm ist nur eine Hilfestellung, im Zweifel gilt die Textform.\\
\tikzstyle{fffnode} = [rectangle, minimum width=2cm, minimum height=0.5cm,text width=3cm, text centered, draw=white, fill=fffdgreen, text=white]
\tikzstyle{fffnodeabstract} = [rectangle, minimum width=2cm, minimum height=0.5cm,text width=3cm, text centered, draw=white, fill=ffflgreen, text=black]
\tikzstyle{arrow} = [thick,->,>=stealth]
\tikzstyle{arrowb} = [thick,<->,>=stealth]
\tikzstyle{arrowabstract} = [thick,->,>=stealth, dashed]
\tikzstyle{arrowabstractb} = [thick,<->,>=stealth, dashed]

\begin{tikzpicture}[node distance=3cm]

      \node (deligruppe) [fffnode] {BW-Deli Info};
      \node (awareness) [fffnode, right of=deligruppe, xshift=3.5cm] {Awareness- beauftragte};
      \node (bwlg) [fffnode, right of=awareness, xshift=3.5cm] {FFF BW LG};
      \node (deli) [fffnode, below of=deligruppe] {Delegierte};
      \node (og) [fffnode, below of=deli] {Ortsgruppen};
      \node (prozess) [fffnode, right of=og, xshift=1 cm] {ProzessLG};
      \node (content) [fffnode, right of=prozess] {ContentLG};
      \node (oea) [fffnode, right of=content, xshift=3cm] {ÖALG};
      \node (output) [fffnode, below of=oea] {Öffentlichkeit};
      \node (input) [fffnode, below of=content] {Kontakte in die Politik / zu Experten};
      \node (lg) [fffnodeabstract, above of=oea] {LGs};


      \draw [arrow] ([xshift=0.5cm]og.north) -- node[anchor=west] {wählt} ([xshift=0.5 cm]deli.south);
      \draw [arrow] (deli) -- node[anchor=east] {informieren} (og);
      \draw [arrow] (deli.west)--([shift={(-0.5cm,0cm)}]deli.west)--([shift={(-0.5cm,-3.8cm)}]deli.west) -- node[anchor=north] {reichen Abstimmungen ein} ([shift={(-0.5cm,-0.5cm)}]prozess.south)--([shift={(-0.5cm,0cm)}]prozess.south);
      \draw [arrow] (awareness) -- node[anchor=south] {beraten} (bwlg);
      \draw [arrow] (awareness) -- node[anchor=south] {beraten} (deligruppe);
      \draw [arrow] (deli) -- node[anchor=south, rotate=23] {wählt} (awareness.south);
      \draw [arrow] ([xshift=0.5 cm]deli.north) -- node[anchor=west] {stimmt ab} ([xshift=0.5 cm]deligruppe.south);
      \draw [arrow] (deligruppe) -- node[anchor=east] {informiert} (deli);
      \draw [arrowb] (og) -- node[anchor=south,rotate=23] {Austausch} ([xshift=-0.3 cm]bwlg.south);
      \draw [arrowb] (content) -- node[anchor=east] {Austausch} (input);
      \draw [arrow] (oea) -- node[anchor=west] {informieren} (output);
      \draw [arrowb] (content) --node[anchor=south] {absprechen} (oea);
      \draw [arrow] (prozess.north) --node[anchor=south,rotate=33] {moderieren} ([xshift=-0.15 cm]bwlg.south);
      \draw [arrow] (prozess.south)--([shift={(0cm,-1.4cm)}]prozess.south) --node[anchor=north] {erstellen Abstimmungen, informieren} ([shift={(-0.8cm,-7.7cm)}]deligruppe.west)--([shift={(-0.8cm,0cm)}]deligruppe.west)--(deligruppe.west);
      \draw [arrowabstract] (prozess) -- ([xshift=-0.3 cm]lg.south);
      \draw [arrowabstract] (content.north) -- ([xshift=-0.15 cm]lg.south);
      \draw [arrowabstract] (oea) -- (lg);
      \draw [arrowb] (bwlg) --node[anchor=west]{Austausch} (lg);

      \draw [arrow] (lg.west) -- node[anchor=south, rotate =23] {reichen Abstimmungen ein} (prozess);












\end{tikzpicture}


\section{Aktionskonsens und Selbstverständnis}
\begin{enumerate}
      \item Wir halten uns an die Definition und das Selbstverständnis des bundesweiten Strukturpapiers unter \href{https://fffutu.re/strupa}{https://fffutu.re/strupa}
      \item Falls sich eine Ortsgruppe nicht an das Selbstverständnis hält, kann sie über die Bundesebene
            ausgeschlossen werden. Daher müssen Ortsgruppen die auf BW-Ebene aktiv sein wollen, auch
            legitime Ortsgruppen der Bundesebene sein.
      \item Alle Wahlen und Abstimmung müssen demokratisch und transparent sein.
      \item Eine Wahl ist in diesem Sinne demokratisch, wenn sie allgemein (hinsichtlich aller
            Stimmberechtigten), frei und gleich ist.
\end{enumerate}
\section{Ortsgruppen (OGs)}
\begin{enumerate}
      \item Die OGs sind selbstständig organisiert und von der Landesebene unabhängig.
      \item Die Landesebene ist abhängig von den Abstimmungen der OGs und kann von diesen ermächtigt und befugt werden, in ihrem Namen zu handeln.
      \item Die OGs handeln eigenverantwortlich und sind demokratisch.
      \item Es können von außen (Landesebene) ausschließlich Tipps und Handlungsempfehlungen gegeben
            werden. Die Landesebene hat allerdings keinerlei Weisungsbefugnis oder Autorität über die
            Ortsgruppen. Es sei denn, es fanden verbindliche Zusagen statt.
      \item Jede Ortsgruppe definiert die Mitglieder ihres Orga-Teams selbst. Diese sollten vertrauenswürdig
            sein und keine vertraulichen Informationen veröffentlichen.
\end{enumerate}
\section{Delis}
\begin{enumerate}
      \item Jede Ortsgruppe kann bis zu 2 Vertreter*innen als Delegierte in eine \hyperlink{Chatstruktur}{landesweite Gruppe}
            entsenden, welche zur Kommunikation von Informationen zu den Ortsgruppen dient. Diese
            Vertreter*innen müssen in allen Abstimmungen die Meinung ihrer Ortsgruppe wiedergeben.
      \item Dabei sollte wenn möglich auf ein ausgewogenes Geschlechterverhältnis geachtet werden.
      \item Die Wahl der Vertreter*innen der Ortsgruppen muss demokratisch sein. Das heißt alle Orga-Team
            Mitglieder müssen theoretisch die Möglichkeit zur Wahl haben. Delegierte müssen die Wahl durch
            ihre Ortsgruppe bestätigen können, etwa durch ein Protokoll oder eine Nachricht der jeweiligen
            Ortsgruppe. Ab dem Zeitpunkt, ab dem eine Ortsgruppe die Landesebene über die Abwahl einer
            Person informiert, ist diese nicht mehr delegiert, und wird aus der \glqq{}BW-Deli Info\grqq-
            Gruppe entfernt.
      \item Ortsgruppen können ihre Vertreter*innen frei bestimmen. Es ist möglich, dass gewählte
            Vertreter*innen auch auf weiteren Ebenen aktiv sind, oder delegiert werden.

\end{enumerate}
\section{Abstimmungen}
Delegierte reichen angefragte Entscheidungen an die OG weiter. In der OG wird über die Fragestellung
entschieden. In der Abstimmung gibt die*der Delegierte die Meinung der OG wieder.
\subsection{Allgemeines}
\begin{enumerate}
      \item Jede OG hat eine Stimme, also verfügt jede OG über gleiche Stimmkraft. Hat eine Ortsgruppe
            (versehentlich) mehrere Stimmen abgegeben, so wird die zuletzt abgegebene Stimme gewertet.
      \item  Jede OG und Landesgruppe darf eine Abstimmung initiieren.
      \item Besteht Klärungsbedarf, kann innerhalb der Abstimmungsfrist eine Telefonkonferenz einberufen
            werden.
\end{enumerate}
\subsection{Ortsgruppen-Abstimmung}
\begin{enumerate}
      \item Abstimmungsformulare werden von der Prozess-LG (üblicherweise deren Ansprechpersonen, kann
            aber innerhalb der LG abgewichen werden) erstellt und durchgeführt.
            \begin{enumerate}
                  \item Alle Informationen und Möglichkeiten zu Abstimmungen werden immer in die \glqq{}BW-Deli Info\grqq-gruppe gestellt. Dabei ist die Abstimmungsfrist, der Link zum Formular
                        sowie mindestens eine Ansprechperson zu nennen.
                  \item Aus Transparenzgründen werden die Abstimmungen zusätzlich, ohne Link zum
                        Abstimmungsformular, in die Gruppe \glqq{}FFF Baden-Württemberg LG\grqq und nach Bedarf und
                        Möglichkeiten in weitere Infogruppen gestellt.
            \end{enumerate}

      \item Jeder Abstimmung müssen mindestens 10 Tage Zeit eingeräumt werden, es sei denn eine
            Ortsgruppe oder Landesarbeitsgruppe beantragt eine dringende Abstimmung. Die Frist beginnt
            sofort mit der Veröffentlichung des Antrags in Deli-Gruppen.
      \item Ein Antrag darf nach Beginn der Abstimmung nicht mehr verändert werden, allerdings kann er
            jederzeit zurückgezogen werden.
      \item Für ein positives Ergebnis der Abstimmung ist eine einfache Mehrheit nötig \(>50\%\). Dabei werden
            Enthaltungen nicht berücksichtigt.
      \item Eine Liste mit den Stimmen aller OGs ist nach der Abstimmung durch die Prozess-LG
            in den Gruppen \glqq{}FFF Baden-Württemberg LG\grqq und \glqq{}BW-Deli Info\grqq zu veröffentlichen.
\end{enumerate}
\subsection{dringende Abstimmung}
\begin{enumerate}
      \item Eine dringende Abstimmung ist nur anzuwenden, wenn durch einen regulären
            Abstimmungsprozess der Fridays for Future Bewegung Schaden zugefügt wird oder das Thema der
            Abstimmung wichtig ist und eine kurzfristige Abstimmung erforderlich macht.
      \item Der Zeitrahmen, der für eine dringende Abstimmung zur Verfügung steht, wird von
            der*dem Antragsstellenden festgelegt, beträgt jedoch mindestens 72 Stunden.
      \item Mit besonderer Begründung können 72 Stunden unterschritten werden. Dabei
            benötigt die Abstimmung eine einfache 2/3 Mehrheit sowie mindestens 2/3 der
            beteiligten Ortsgruppen im Durchschnitt der letzten 5 Abstimmungen (Quorum)
      \item Bei jeder dringenden Abstimmung gibt es die Nachfrage, ob die Abstimmung für
            dringend erachtet wird. Sollte dieser Punkt nicht mit
            einer einfachen Mehrheit(> 50 \%) angenommen werden, wird ein reguläres Verfahren
            durchgeführt. Diese Option kann zusätzlich zur normalen Stimme ausgewählt werden.
\end{enumerate}
\subsection{Aufschiebendes Vetorecht}
\begin{enumerate}
      \item Ein Veto muss immer begründet sein. Diese Begründung muss konstruktiv und somit konfliktlösend
            sein und schon bei der Einlegung bekannt gemacht werden.
            Ein Veto heißt: \glqq{}Ich kann diese Entscheidung nicht mittragen und lege formalen Protest ein\grqq
      \item Das Veto ist ein aufschiebendes Veto, es kann von jeder OG gestellt werden. Es ist als Ultima Ratio
            anzusehen.
            \begin{enumerate}
                  \item Aufschiebend bedeutet: \glqq{}Wir müssen nochmal über die Entscheidung diskutieren\grqq
            \end{enumerate}
      \item Das Veto muss innerhalb der Rückmeldefrist für reguläre Abstimmungen zusammen mit einem
            konstruktiven Feedback, im Abstimmungsformular eingereicht werden.
      \item Das konstruktive Feedback (= Begründung des Vetos) muss von einer einfachen 1⁄3 Mehrheit
            Zustimmung in der darauf folgenden Abstimmung über die Zulässigkeit des Vetos angenommen
            werden. Für diese Abstimmung sind 72 Stunden Zeit einzuräumen.
            \begin{enumerate}

                  \item Bei positivem Ausgang muss der dem Veto zugrunde liegende Entwurf überarbeitet
                        werden. Für die Kompromissfindung sind vier Tage angesetzt.
                  \item Die Kompromissfindung wird durch Schlichter*innen unterstützt. Die Schlichtung soll von
                        geeigneten, am Konflikt unbeteiligten Menschen übernommen werden. Wenn sich die
                        Konfliktparteien nicht gemeinsam auf eine(n) Schlichter*in einigen können, können sie
                        sich an die Prozess-LG wenden, die dann eine vermittelnde Rolle
                        einnehmen kann.
                        \begin{enumerate}
                              \item Wird ein Kompromiss erzielt und das Veto aufgehoben, wird dieser überarbeitete
                                    Entwurf zur Abstimmung gestellt. Für die Annahme ist eine einfache Mehrheit
                                    notwendig.
                              \item Ist es zu keiner Einigung gekommen, wird der ursprüngliche Entwurf zur
                                    Abstimmung gestellt. Für die Annahme ist eine einfache 3⁄4 Mehrheit notwendig.
                              \item Diese Abstimmungen sind über 72 Stunden anzusetzen. Scheitern diese so gilt
                                    der Entwurf als endgültig abgelehnt.
                        \end{enumerate}
            \end{enumerate}
      \item Bei negativem Ausgang der Abstimmung, unabhängig möglicher Vetos, wird der Entwurf
            verworfen.
\end{enumerate}
\subsection{Welche Mehrheiten und Besonderheiten haben wir?}
\begin{enumerate}
      \item Für Änderungen an der Struktur ist eine ⅔-Mehrheit notwendig.
      \item Bei Personenwahlen kommen besondere Regelungen zum tragen, diese stimmen mit denen des
            vergleichbaren Gremiums der Bundesebene überein.
            \begin{enumerate}
                  \item Abweichend davon ist immer eine Zustimmung von mindestens 50 \% notwendig.
                  \item Auch soll in einer zukünftigen Version bei ausreichender Menschenanzahl angestrebt
                        werden, dass sich die Ämter in der der LG-Ansprechpartner*innen und der
                        Awarenessbeauftragten gegenseitig Ausschließen, heißt das man nur in einem der
                        Positionen gleichzeitig sein kann.
            \end{enumerate}
\end{enumerate}
\subsection{Abstimmungsformulare}
\begin{enumerate}
      \item Diese Informationen müssen im Abstimmungsformular festgeschrieben stehen:
            \begin{enumerate}
                  \item Ende der Abstimmungsfrist
                  \item Antrag in voller Länge oder Link zu einem (read-only)Dokument mit dem Antragstext.
            \end{enumerate}
      \item Diese Informationen müssen abgefragt werden:
            \begin{enumerate}
                  \item Name
                  \item Ortsgruppe
                  \item Kontaktdaten(z.B. Handynummer oder Mailadresse)
                  \item \glqq{}Bist du ein Landesdelegierter aus einer OG in BW?\grqq( Ja / Nein )
                  \item Abstimmungsverhalten (Ja / Nein / Enthaltung)
                  \item Bei dringenden Abstimmungen: \glqq{}Findest du die Abstimmung dringend?\grqq
                        (Ja / Nein / Enthaltung)
                  \item \glqq{}Willst du ein Veto einlegen?\grqq (Ja / Nein )
                  \item Falls die Antwort ja sei, konstruktive Begründung des Vetos
                  \item Ggf. Feedback zum Antrag
            \end{enumerate}
\end{enumerate}
\section{Landesarbeitsgruppen(LGs)}
\begin{enumerate}
      \item
            Landesarbeitsgruppen sind Arbeitsgruppen auf Landesebene, welche sich mit einem bestimmten Thema
            befassen. Hierbei gibt es drei stetige Landesarbeitsgruppen (Handlungsspielraum unten), darüber hinaus
            können weitere per Ortsgruppenabstimmung temporär legitimiert werden.
      \item
            Diese müssen einen
            Handlungsspielraum definieren und bei Zweifel ihre Arbeit mit den bestehenden LGs absprechen.
      \item
            In LGs
            kann jeder mitarbeiten, zur Auflösung von weiteren LGs ist eine 2/3 Mehrheit notwendig.
      \item
            LGs stellen
            gegenüber der Landesebene Ansprechpersonen zur Verfügung.
      \item
            Landesarbeitsgruppen sind zudem angehalten zusätzlich zu dem Handlungsspielraum, intern ihre
            Arbeitsweise zu Regeln und Festzuhalten. Hiervon können LGs selbstständig begründet abweichen,
            allerdings muss dies intern über mindestens einen Tag abgestimmt oder in einer TK beschlossen werden.
            Ebenso können Änderungen beschlossen werden.
      \item
            LGs können nach eigenem ermessen Untergruppen gründen.
\end{enumerate}

\subsection{Prozess-LG}
\begin{enumerate}
      \item Die Prozess-LG kümmert sich um das Erstellen und Durchführen der Abstimmungen auf
            Landesebene, die Weiterentwicklung der Struktur, das Pflegen der Kontaktdaten sowie die
            Transparenz der Landesebene gegenüber den Ortsgruppen.
      \item Da es sich bei den Delegierten-Kontakten und Adminrechte um sensible Informationen handelt,
            kümmern sich hierum die LG Ansprechpersonen.
      \item Es sollen regelmäßig, ca. monatlich Infonachrichten über die Arbeit der Landesebene an die
            Ortsgruppen gesendet werden.
      \item Personen, welche am Inhalt einer Abstimmung beteiligt sind, dürfen nicht bei deren Durchführung
            und Auswertung beteiligt sein. Des Weiteren bekommt neben der durchführenden Person eine
            weitere Person zur Kontrolle Zugang zum Abstimmungsformular
\end{enumerate}
\subsection{Content-LG}
\begin{enumerate}
      \item Die Content-LG erstellt Forderungsdokumente, Führt Gespräche mit Menschen / Organisationen
            an die sich die Forderungen Richten sowie plant Kampagnen auf Landesebene.
      \item Forderungsdokumente müssen von den Ortsgruppen abgestimmt werden. Sie sollen anschließend
            auch weiterverfolgt und in die Politiker*innengespräche einfließen.
      \item Eine Kampagene soll über einen Zeitraum einen Fokus auf ein Thema lenken.
      \item Die Öffentlichkeitsarbeit hierzu übernimmt die Öffentlichkeitsarbeits-LG.
      \item Die Content-LG kann die Unterstützung von Externen Forderungen auf BW-Ebene
            beschließen(Logoverwendung). Im Zweifel ist eine Ortsgruppen-Abstimmung durchzuführen.
\end{enumerate}
\subsection{Öffentlichkeitsarbeits-LG}
\begin{enumerate}
      \item Die Öffentlichkeitsarbeit-LG kümmert sich auf Landesebene um die Öffentlichkeitsarbeit. Darunter
            Fallen insbesondere: Öffentlichkeitsarbeit auf Social Media, Öffentlichkeitsarbeit auf der
            Webseite, sowie Pressearbeit und das bearbeiten / verteilen eingehender E-Mails.
      \item Die Öffentlichkeitsarbeit-LG ist legitimiert, selbst Inhalte zu Veröffentlichen, welche von der
            Bewegung stammen. Sie ist nicht Legitimiert eigene Forderungen aufzustellen, hierfür ist die
            Content-LG zuständig, welche die Öffentlichkeitsarbeit-LG weiterverbreitet.
      \item SocialMedia umfasst das erstellen von Inhalten für Dienste wie Facebook, Twitter, Instagram,
            Youtube, etc.
      \item Pressenarbeit umfasst die Kommunikation sowie das Abgeben von Statements an Medien,
            beispielsweise durch Beantwortung von Anfragen der Presse, versenden von Pressemitteilungen
            sowie organisation von Pressekonferenzen.
      \item Ziel hierbei ist es, die Bewegung in der Öffentlichkeit zu repräsentieren und die Forderungen der
            Bewegung zu verbreiten, sowie zu Veranstaltungen aufzurufen.
      \item Die Öffentlichkeitsarbeits-LG ist legitmiert, über die Logoverwendung auch für Externe
            Kooperationen zu Netscheiden. Dies erfolgt mit einer 2/3 Mehrheit. Falls es um Forderungen
            geht entscheidet die Content-LG. Bei Zweifel ist eine Ortsgruppenentscheidung durchzuführen.
      \item Die Öffentlichkeitsarbeits-LG soll die Ortsgruppen bei deren Öffentlichkeitsarbeit unterstützen.
\end{enumerate}
\subsection{Mailadresse}
\begin{enumerate}
      \item Auf Landesebene gibt es eine Mailadresse, sie wird von den LGs gemeinsam genutzt.
      \item Jede LG bestimmt bis zu 8 Personen für den Mailzugriff, diese können Mails versenden und deren
            Antworten bearbeiten.
      \item Für die Zuteilung eingehender Mails ist die Öffentlichkeitsarbeits-LG zuständig.
      \item Diese haben zur Absprache eine eigene Gruppe. Kontaktdaten aus Mails dürfen nicht ohne
            Zustimmung weitergegeben werden.
\end{enumerate}
\section{Chatstruktur}
\subsection{BW-Deli Info}
\begin{enumerate}
      \item Mitglieder dieser Gruppe sind die Delis, die Ansprechpersonen für die LGs und die Awarenessbeauftragten.
      \item Hier werden Abstimmungen mit Abstimmungslink, deren Ergebnisse, mögliche Vetos sowie
            weitere für die Ortsgruppen relevante Informationen geteilt.
      \item Nur die Ansprechpersonen der Prozess-LG haben Schreibrechte.
\end{enumerate}
\subsection{FFF Baden-Württemberg LG}
\begin{enumerate}
      \item Sie steht jedem Orga-Team Mitglied aus einer Ortsgruppe aus Baden-Württemberg und jedem
            Mitglied einer Landesarbeitsgruppe von Baden-Württemberg offen.
      \item Hier werden Abstimmungen ohne Abstimmungslink, deren Ergebnisse, mögliche Vetos sowie
            weitere relevante Infos geteilt.
      \item Jeder hat Schreibrechte.
\end{enumerate}
\section{Antidiskriminierung und Awareness}
In der Zusammenarbeit wird auf gegenseitige Rücksichtnahme und einen wertschätzenden konstruktiven
Umgang geachtet.
\subsection{Awareness-Beauftragte}
\begin{enumerate}
      \item Die Delegierten wählen 2 Awareness-Beauftragte. Es ist nicht gestattet, dass mehr als die Hälfte
            der Awareness-Beauftragten desselben Geschlechts sind.
            \begin{enumerate}
                  \item Die Awareness-Beauftragten benötigen eine 2/3 Mehrheit um gewählt zu werden, und
                        können mit mindestens 1/3 der Stimmen abgewählt werden.
            \end{enumerate}
      \item Die Awareness-Beauftragten sind Ansprechpersonen, sollten Personen sich unwohl fühlen oder
            diskriminiert werden.
      \item Awareness-Beauftragte können, müssen aber nicht von einer Ortsgruppe delegiert sein.
      \item Awareness-Beauftragte können jederzeit die Delegierten oder die Landesarbeitsgruppen über
            Antidiskriminierungsthemen informieren, und Stellung beziehen, sollten Entscheidungen getroffen
            werden, die diskriminierend oder problematisch sind.
\end{enumerate}
\subsection{Ausschlussgremium}
\begin{enumerate}
      \item Aus der Landesebene können durch das bundesweite Ausschlussgremium bei Fällen von
            Diskriminierung und / oder übergriffigem Verhalten Maßnahmen nach deren Handlungsspielraum
            ergriffen werden.
\end{enumerate}

\end{document}